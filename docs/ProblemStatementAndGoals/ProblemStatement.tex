\documentclass{article}

\usepackage{tabularx}
\usepackage{booktabs}

\title{Problem Statement and Goals\\\progname}

\author{\authname}

\date{}

%% Comments

\usepackage{color}

\newif\ifcomments\commentstrue %displays comments
%\newif\ifcomments\commentsfalse %so that comments do not display

\ifcomments
\newcommand{\authornote}[3]{\textcolor{#1}{[#3 ---#2]}}
\newcommand{\todo}[1]{\textcolor{red}{[TODO: #1]}}
\else
\newcommand{\authornote}[3]{}
\newcommand{\todo}[1]{}
\fi

\newcommand{\wss}[1]{\authornote{magenta}{SS}{#1}} 
\newcommand{\plt}[1]{\authornote{cyan}{TPLT}{#1}} %For explanation of the template
\newcommand{\an}[1]{\authornote{cyan}{Author}{#1}}

%% Common Parts

\newcommand{\progname}{Software Engineering} % PUT YOUR PROGRAM NAME HERE
\newcommand{\authname}{Team 24, Cowgnition
\\ Sylvia Kamel
\\ Rosa Chen
\\ Karim Elbasiouni
\\ Claire Nielsen
\\ Safwan Khan} % AUTHOR NAMES                  

\usepackage{hyperref}
    \hypersetup{colorlinks=true, linkcolor=blue, citecolor=blue, filecolor=blue,
                urlcolor=blue, unicode=false}
    \urlstyle{same}
                                


\begin{document}

\maketitle

\begin{table}[hp]
\caption{Revision History} \label{TblRevisionHistory}
\begin{tabularx}{\textwidth}{llX}
\toprule
\textbf{Date} & \textbf{Developer(s)} & \textbf{Change}\\
\midrule
Date1 & Name(s) & Description of changes\\
Date2 & Name(s) & Description of changes\\
... & ... & ...\\
\bottomrule
\end{tabularx}
\end{table}

\section{Problem Statement}

\subsection{Problem}
Dairy and cattle farmers are responsible for the health and wellness of thousands of cows. The condition of the livestock in their care directly influences profit margins and productivity of the farm as a whole. Monitoring health conditions and behaviour must be done to minimize consequences of illness and injury, but the process is labour intensive and time consuming, making it difficult to track animal health, infections, weight, and productivity effectively. Not only do any minuscule changes need to be noticed within dozens of cows, the cows then need to be identified often by the farmer walking through the pen to check an ear tag. The length and complexity of the process make monitoring cow wellness incredibly difficult. The project aims to provide a solution that simplifies the monitoring process and provides an early indication of health issues, improving animal welfare and therefore farm productivity. \\ 

\subsection{Inputs and Outputs}
\begin{itemize}
    \item Input: Video feed showing cows where they live at a farm. Setting to be determined.
    \item Output: User readable data reporting cow attributes and/or behaviour. 
\end{itemize} 


\subsection{Stakeholders}
\subsubsection{Direct Stakeholders}
\begin{enumerate}
    \item \textbf{Farmers:} They are the primary users of this software as this project aims to improve the efficiency of their workflows.
    \item \textbf{Luke Schuurman (Supervisor and CATTLEytics Inc. Representative):} Luke has a vested interest in the development of this project. He will
    play a crucial role directing our team throughout the project. He will be the main point of contact to garner a shared understanding of the vision CATTLEytics Inc. has
    for this project, and as such he will provide us with the necessary resources, including video data, to ensure the success of this project.
\end{enumerate}

\subsubsection{Indirect Stakeholders}
\begin{enumerate}
    \item \textbf{Food and Agriculture Regulatory Bodies:} They establish rules on animal welfare, food safety and traceability that influence the design and operation of our product.
    \item \textbf{Dairy and Meat Processing Companies, QA Division:} They rely on farms to provide the necessary raw materials that are compliant with safety standards.
\end{enumerate}


\subsection{Environment}
\subsubsection{Software Environment}
Software required for this system to work as expected:
\begin{itemize}
    \item Cloud-based environment to train model (Google Colab, AWS etc.)
    \item Web-based environment to show alerts and results
\end{itemize}

\subsubsection{Hardware Environment}
Hardware required for this system to work as expected:
\begin{itemize}
    \item High-resolution IP or Smart IoT-based camera systems with edge compute devices installed in the farm
    \item Devices for farmers to access any real-time alerts or results of the model
    \item On-premise server to archive videos
\end{itemize}
\textbf{NOTE:} This will be provided and installed by CATTLEytics. Inc in the farms. This is the infrastructure that the system will use to operate.


\section{Goals}

\section{Stretch Goals}
\begin{enumerate}
    \item Build a user interface (e.g. a dashboard) that displays attributes and behaviours of each
    cow in the group and displays aggregated information about the entire group of cows
    \item The system takes in video feed from more than one device
    \item The system allows videos from smart glasses to be an input
    \item The system runs in an offline environment
\end{enumerate}

\section{Extras}

The group will develop a user instructional video and a user manual as supplementary extra materials.  These extra resources will provide comprehensive support to farmers, ensuring they can effectively operate the proposed system to monitor their cow livestock. As farmers are typically middle aged, there may be a learning gap with technology. It is important that our primary stakeholders are supported and the tool is easily usable by them. Approval of the extras will be confirmed with the project supervisor.

\newpage{}

\section*{Appendix --- Reflection}

\textbf{Claire Nielsen, Karim Elbasiouni, Rosa Chen}
\begin{enumerate}
    \item \textit{What went well while writing this deliverable?}\\
    For a first deliverable, the team collaborated well. Since the project is still so broad the team was able to brainstorm together how to best approach a solution to our problem statement. The team was able to schedule meetings to work together with minimal conflict, and the collaborative periods helped make sure all members unanimously agreed and had a good idea of how to proceed. 
    \item \textit{What pain points did you experience during this deliverable, and how did you resolve them?}\\
    One minor obstacle the team hit was the meeting schedule. This week the team was able to be flexible about meeting times, but when attempting to fill out a poll with availability for a recurring meeting there didn’t appear to be a time that worked for everyone. The team was able to decide on a meeting slot and willingness to be flexible in the future. Another problem the team had was the late feedback from our supervisor. There were some questions to be clarified, and a response was not received until late on the due date, putting a time crunch on some sections of the document. 
    \item \textit{How did you and your team adjust the scope of your goals to ensure they are suitable for a Capstone project (not overly ambitious but also of appropriate complexity for a senior design project)?}\\
    The team collaborated on the problem statement and goals for the project bearing in mind that they should be abstract, and not overly ambitious. While the team already has ideas of goals for the system and how exactly to solve the stated problem, they are not included. These documents are intended to be an early stage description of the problem, not a detailed plan for the solution. By keeping the goals and problem statement abstract, there should be fewer design changes later in development. The goals chosen accurately represent a minimum viable product that solves the problem statement, without detail as to how. These goals should be met to produce an adequate solution, and there is room to extend to meet the stretch goals if time and resources permit.
\end{enumerate}

\textbf{Safwan Khan}
\begin{enumerate}
    \item \textit{What went well while writing this deliverable?}\\
     When we were writing points for each section in this deliverable, it felt like we were mostly on the same page and similarly viewed most aspects. As a result, this made it easier for us to write points and make progress towards completing this deliverable. Moreover, as we were discussing and writing points for each section
     in this deliverable, it helped me to continue increasing my understanding of the project, especially the core problem and the system's environment.
     
    \item \textit{What pain points did you experience during this deliverable, and how
    did you resolve them?}\\
    My pain point was trying to just understand the system's hardware and software environment, particularly 
    internet connectivity. This was resolved by asking meaningful follow up questions and verbally confirming my understanding.

    \item \textit{How did you and your team adjust the scope of your goals to ensure
    they are suitable for a Capstone project (not overly ambitious but also of
    appropriate complexity for a senior design project)?}\\
    Collaborated with Claire and Karim. 
\end{enumerate}  

\end{document}