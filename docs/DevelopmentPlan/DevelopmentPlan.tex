\documentclass{article}

\usepackage{booktabs}
\usepackage{tabularx}
\usepackage{float}

\title{Development Plan\\\progname}

\author{\authname}

\date{}

\input{../Comments}
%% Common Parts

\newcommand{\progname}{Software Engineering} % PUT YOUR PROGRAM NAME HERE
\newcommand{\authname}{Team 24, Cowgnition
\\ Sylvia Kamel
\\ Rosa Chen
\\ Karim Elbasiouni
\\ Claire Nielsen
\\ Safwan Khan} % AUTHOR NAMES                  

\usepackage{hyperref}
    \hypersetup{colorlinks=true, linkcolor=blue, citecolor=blue, filecolor=blue,
                urlcolor=blue, unicode=false}
    \urlstyle{same}
                                

\begin{document}

\maketitle

\begin{table}[hp]
\caption{Revision History} \label{TblRevisionHistory}
\begin{tabularx}{\textwidth}{llX}
\toprule
\textbf{Date} & \textbf{Developer(s)} & \textbf{Change}\\
\midrule
Date1 & Name(s) & Description of changes\\
Date2 & Name(s) & Description of changes\\
... & ... & ...\\
\bottomrule
\end{tabularx}
\end{table}

\newpage{}

\wss{Put your introductory blurb here.  Often the blurb is a brief roadmap of
what is contained in the report.}

\wss{Additional information on the development plan can be found in the
\href{https://gitlab.cas.mcmaster.ca/courses/capstone/-/blob/main/Lectures/L02b_POCAndDevPlan/POCAndDevPlan.pdf?ref_type=heads}
{lecture slides}.}

\section{Confidential Information?}

\wss{State whether your project has confidential information from industry, or
not.  If there is confidential information, point to the agreement you have in
place.}

\wss{For most teams this section will just state that there is no confidential
information to protect.}
\section{IP to Protect}

\wss{State whether there is IP to protect.  If there is, point to the agreement.
All students who are working on a project that requires an IP agreement are also
required to sign the ``Intellectual Property Guide Acknowledgement.''}

\section{Copyright License}

\wss{What copyright license is your team adopting.  Point to the license in your
repo.}

\section{Team Meeting Plan}

\wss{How often will you meet? where?}

\wss{If the meeting is a physical location (not virtual), out of an abundance of
caution for safety reasons you shouldn't put the location online}

\wss{How often will you meet with your industry advisor?  when?  where?}

\wss{Will meetings be virtual?  At least some meetings should likely be
in-person.}

\wss{How will the meetings be structured?  There should be a chair for all meetings.  There should be an agenda for all meetings.}

\section{Team Communication Plan}

\wss{Issues on GitHub should be part of your communication plan.}

\section{Team Member Roles}

\wss{You should identify the types of roles you anticipate, like notetaker,
leader, meeting chair, reviewer.  Assigning specific people to those roles is
not necessary at this stage.  In a student team the role of the individuals will
likely change throughout the year.}

\section{Workflow Plan}

\begin{itemize}
	\item How will you be using git, including branches, pull request, etc.?
	\item How will you be managing issues, including template issues, issue
	classification, etc.?
  \item Use of CI/CD
\end{itemize}

\section{Project Decomposition and Scheduling}

\subsection{Github Projects}
Github Projects will be used for project management of tasks. The group will use the Kanban board feature on Github projects to assign tasks. Tasks will be categorized under milestones and will be assigned to individual team members. They can take on one of three states: 
\begin{itemize}
  \item \underline{To do:} Tasks that are backlogged and are not currently being worked on. 
  \item \underline{In progress:} Tasks that are currently being worked on.
  \item \underline{Done:} Completed tasks that have been reviewed by another member and approved.
\end{itemize}

\noindent Github Issues will also be used to track team meeting agendas, supervisor meetings, lecture notes, TA meetings, and peer reviews. The link to the Github repository where the project can be found will be linked \underline{\href{https://github.com/rosachen3/Cowgnition/tree/main}{here}}.

\subsection{Schedule}
The project will follow the major deadlines outlined below. Each task will have subtasks assigned as the semester progresses. These can be tracked in the Github Projects Kanban Board.

\begin{table}[H]

\centering
\caption{List of Milestone Deadlines}
\begin{tabularx}{\textwidth}{|l|X|}
\hline
\textbf{Milestone} & \textbf{Due Date} \\
\hline
Team formation and Project Selection & September 15 \\
\hline
Problem Statement, Goals and Development Plan & September 22 \\
\hline
SRS + HA & October 6 \\
\hline
V\&V Plan Revision 0 & October 27 \\
\hline
Design Documentation Revision -1 & November 10 \\
\hline
Proof of Concept Demonstration & November 17 \\
\hline
Design Document Revision 0 & January 19 \\
\hline
Revision 0 Presentation & February 2 \\
\hline
V and V Report and Extras Revision 0 & March 9 \\
\hline
Final Demonstration (Revision 1) & March 23 \\
\hline
Final Documentation (Revision 1) & April 6 \\
\hline
EXPO Demonstration & TBD \\
\hline
\end{tabularx}
\end{table}

\section{Proof of Concept Demonstration Plan}

CATTLEytics will be providing video training data from local partnering farmers. The videos will be used to analyze and categorize cows based on behaviours and physical traits.  

\vspace{1em}

\noindent The following risks have been identified to evaluate the feasibility of the project. 

\vspace{1em}

\begin{itemize}
  \item \noindent \textbf{Farmer Partnerships:} CATTLEytics is a small startup company. Currently, there is no video training data as resources are provided as needed when developing new features. There are local farmers who have agreed to send video data, but there is a risk of delays which could slow down development. 
  \item \noindent \textbf{Data Quality and Quantity:} Machine learning algorithms depend heavily on the data set. There is a risk that the video data collected from farmers will be of low quality or quantity, reducing the accuracy and reliability of the machine learning model. For the Proof of Concept, this risk can be mitigated by using publicly available video data of cows. 
  \item \noindent \textbf{GPU Resources:} The CAS department at McMaster offers departmental resources for machine learning purposes. As these resources are limited, there is a risk that these resources may not be available when the model needs to be trained. A possible risk mitigation strategy is to leverage the use of cloud-based GPU resources to supplement departmental resources if needed.
  \item \noindent \textbf{Environmental Requirements:} Farmers in rural communities may have limited hardware and connectivity. If a system is too resource intensive, farmers may face challenges using the system effectively. For the Proof of Concept, this risk can be mitigated by demonstrating the usage of processing and analyzing collected video data in a controlled environment, without requiring real-time analysis. 
  \item \noindent \textbf{Implementation:} Team members have limited experience with machine learning problems. As a result, researching about existing open-source models and fundamentals may lead to delays. However, the project will be supervised by Luke Schuurman from CATTLEytics who can provide technical advice from a software engineering standpoint. 
  \item \noindent \textbf{Licensing:} Many popular frameworks and pretrained models have licenses that prohibit commercial use. While this may not impact the Proof of Concept, the licenses can become a risk as the project progresses into a commercially available product. This risk can be mitigated by documenting the licenses of all tools and commercially available alternatives.
\end{itemize}

\noindent \textbf{POC Plan:} The Proof of Concept will show that the system is functional and that the risks mentioned can be mitigated. This will ensure that the project can be completed on schedule. 
The demonstration will show that the machine learning model can successfully identify and classify one cow parameter. The demonstration will use a small dataset to validate the system, proving that the model can successfully preprocess and train data using the available GPU resources. The results of the model will be presented in a user-friendly format for farmers to easily access. 

\section{Expected Technology}

\wss{What programming language or languages do you expect to use?  What external
libraries?  What frameworks?  What technologies.  Are there major components of
the implementation that you expect you will implement, despite the existence of
libraries that provide the required functionality.  For projects with machine
learning, will you use pre-trained models, or be training your own model?  }

\wss{The implementation decisions can, and likely will, change over the course
of the project.  The initial documentation should be written in an abstract way;
it should be agnostic of the implementation choices, unless the implementation
choices are project constraints.  However, recording our initial thoughts on
implementation helps understand the challenge level and feasibility of a
project.  It may also help with early identification of areas where project
members will need to augment their training.}

Topics to discuss include the following:

\begin{itemize}
\item Specific programming language
\item Specific libraries
\item Pre-trained models
\item Specific linter tool (if appropriate)
\item Specific unit testing framework
\item Investigation of code coverage measuring tools
\item Specific plans for Continuous Integration (CI), or an explanation that CI
  is not being done
\item Specific performance measuring tools (like Valgrind), if
  appropriate
\item Tools you will likely be using?
\end{itemize}

\wss{git, GitHub and GitHub projects should be part of your technology.}

\section{Coding Standard}

\wss{What coding standard will you adopt?}

\newpage{}

\section*{Appendix --- Reflection}

\wss{Not required for CAS 741}

\input{../Reflection.tex}

\begin{enumerate}
    \item Why is it important to create a development plan prior to starting the
    project?
    \item In your opinion, what are the advantages and disadvantages of using
    CI/CD?
    \item What disagreements did your group have in this deliverable, if any,
    and how did you resolve them?
\end{enumerate}

\newpage{}

\section*{Appendix --- Team Charter}

\wss{borrows from
\href{https://engineering.up.edu/industry_partnerships/files/team-charter.pdf}
{University of Portland Team Charter}}

\subsection*{External Goals}

\wss{What are your team's external goals for this project? These are not the
goals related to the functionality or quality fo the project.  These are the
goals on what the team wishes to achieve with the project.  Potential goals are
to win a prize at the Capstone EXPO, or to have something to talk about in
interviews, or to get an A+, etc.}

\subsection*{Attendance}

\subsubsection*{Expectations}

\begin{itemize}
  \item \textbf{Weekly Meetings}: With the exception of acceptable excuses and emergencies, team members are expected to attend all meetings. This includes team, supervisor, and TA meetings. Attendance will be logged through the usage of Github Issues. 
  \item \textbf{Punctuality}: It is expected that all team members arrive on time to meetings. If a group member is expected to be more than 15 minutes late or require to leave early for a meeting, the group should be notified through the designated Microsoft Teams or SMS group chat as soon as possible. If a team member needs to miss a meeting, the member must notify the group with a valid reason the day before. They are responsible for reading the meeting minutes, and organizing an alternative time individually or with another team member to catch up on missed work.
\end{itemize}

\subsubsection*{Acceptable Excuse}

\wss{What constitutes an acceptable excuse for missing a meeting or a deadline?
What types of excuses will not be considered acceptable?}

\subsubsection*{In Case of Emergency}

\wss{What process will team members follow if they have an emergency and cannot
attend a team meeting or complete their individual work promised for a team
deliverable?}

\subsection*{Accountability and Teamwork}

\subsubsection*{Quality} 

\wss{What are your team's expectations regarding the quality
of team members' preparation for team meetings and the quality of the
deliverables that members bring to the team?}

\subsubsection*{Attitude}

\wss{What are your team's expectations regarding team members' ideas,
interactions with the team, cooperation, attitudes, and anything else regarding
team member contributions?  Do you want to introduce a code of conduct?  Do you
want a conflict resolution plan?  Can adopt existing codes of conduct.}

\subsubsection*{Stay on Track}

\wss{What methods will be used to keep the team on track? How will your team
ensure that members contribute as expected to the team and that the team
performs as expected? How will your team reward members who do well and manage
members whose performance is below expectations?  What are the consequences for
someone not contributing their fair share?}

\wss{You may wish to use the project management metrics collected for the TA and
instructor for this.}

\wss{You can set target metrics for attendance, commits, etc.  What are the
consequences if someone doesn't hit their targets?  Do they need to bring the
coffee to the next team meeting?  Does the team need to make an appointment with
their TA, or the instructor?  Are there incentives for reaching targets early?}

\subsubsection*{Team Building}

\begin{itemize}
  \item The team plans on participating in social activities outside of capstone meetings together. These events will be hosted at least once a semester and will occur either on campus or outside of campus (i.e. food, cafes, etc.).  
\end{itemize}  


\subsubsection*{Decision Making} 

\wss{How will you make decisions in your group? Consensus?  Vote? How will you
handle disagreements? }

\end{document}