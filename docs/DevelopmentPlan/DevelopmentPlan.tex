\documentclass{article}

\usepackage{booktabs}
\usepackage{tabularx}
\usepackage{float}

\title{Development Plan\\\progname}

\author{\authname}

\date{}

%% Comments

\usepackage{color}

\newif\ifcomments\commentstrue %displays comments
%\newif\ifcomments\commentsfalse %so that comments do not display

\ifcomments
\newcommand{\authornote}[3]{\textcolor{#1}{[#3 ---#2]}}
\newcommand{\todo}[1]{\textcolor{red}{[TODO: #1]}}
\else
\newcommand{\authornote}[3]{}
\newcommand{\todo}[1]{}
\fi

\newcommand{\wss}[1]{\authornote{magenta}{SS}{#1}} 
\newcommand{\plt}[1]{\authornote{cyan}{TPLT}{#1}} %For explanation of the template
\newcommand{\an}[1]{\authornote{cyan}{Author}{#1}}

%% Common Parts

\newcommand{\progname}{Software Engineering} % PUT YOUR PROGRAM NAME HERE
\newcommand{\authname}{Team 24, Cowgnition
\\ Sylvia Kamel
\\ Rosa Chen
\\ Karim Elbasiouni
\\ Claire Nielsen
\\ Safwan Khan} % AUTHOR NAMES                  

\usepackage{hyperref}
    \hypersetup{colorlinks=true, linkcolor=blue, citecolor=blue, filecolor=blue,
                urlcolor=blue, unicode=false}
    \urlstyle{same}
                                


\begin{document}

\maketitle

\begin{table}[hp]
\caption{Revision History} \label{TblRevisionHistory}
\begin{tabularx}{\textwidth}{llX}
\toprule
\textbf{Date} & \textbf{Developer(s)} & \textbf{Change}\\
\midrule
Date1 & Name(s) & Description of changes\\
Date2 & Name(s) & Description of changes\\
... & ... & ...\\
\bottomrule
\end{tabularx}
\end{table}

\newpage{}

\wss{Put your introductory blurb here.  Often the blurb is a brief roadmap of
what is contained in the report.}

\wss{Additional information on the development plan can be found in the
\href{https://gitlab.cas.mcmaster.ca/courses/capstone/-/blob/main/Lectures/L02b_POCAndDevPlan/POCAndDevPlan.pdf?ref_type=heads}
{lecture slides}.}

\section{Confidential Information?}

Our project has confidential information from industry. At the moment, we are waiting to recieve a formal non-disclosure agreement from our industry advisor.

\section{IP to Protect}

\wss{State whether there is IP to protect.  If there is, point to the agreement.
All students who are working on a project that requires an IP agreement are also
required to sign the ``Intellectual Property Guide Acknowledgement.''}

\section{Copyright License}

\wss{What copyright license is your team adopting.  Point to the license in your
repo.}

\section{Team Meeting Plan}
\textbf{Team meetings} will be held once a week on Mondays at 6pm for an hour to share general updates and to delegate tasks. The meeting will be moved to an open lecture time slot if and only if a lecture is not scheduled on a given week and the team is available during that lecture slot. Meetings will be held online on Teams if they are scheduled outside of lecture time. Meetings will be held in person if they are scheduled during lecture time. Whether a meeting is scheduled on Monday at 6pm or during a lecture slot will be confirmed on the Sunday before the week starts.

\vspace{1em}

\noindent \textbf{Additional meetings} may be scheduled on demand during empty lecture slots or during a time when everyone required for said meeting is available.

\vspace{1em}

\noindent \textbf{Weekly meetings with the industry advisor} will be held virtually on Teams weekly on an agreed upon day and time. 

\vspace{1em}

\noindent \textbf{Meeting Format:}
The Notetaker will share the meeting agenda in the Teams channel the day before the meeting is scheduled for. The Backlog Manager will create a GitHub Issue for the meeting. Team meetings will be led by the Meeting Chair and will follow the agenda template provided below:
\begin{itemize}
  \item \textbf{10-15 minutes:} Individual updates on personal tasks (updating kanban board on GitHub)
  \item \textbf{40 minutes:} Group work (project deliverables/decision making/brainstorming)
  \item \textbf{5-10 minutes:} Task delegation for the week (updating kanban board on GitHub)
\end{itemize}

\noindent Industry advisor meetings will follow the breakdown provided below:
\begin{itemize}
  \item \textbf{10-15 minutes:} Individual updates on personal tasks and plan for upcoming week
  \item \textbf{30 minutes:} Q\&A with Industry Advisor
\end{itemize}

\section{Team Communication Plan}

\wss{Issues on GitHub should be part of your communication plan.}

\section{Team Member Roles}

\wss{You should identify the types of roles you anticipate, like notetaker,
leader, meeting chair, reviewer.  Assigning specific people to those roles is
not necessary at this stage.  In a student team the role of the individuals will
likely change throughout the year.}

\section{Workflow Plan}

\begin{itemize}
	\item How will you be using git, including branches, pull request, etc.?
	\item How will you be managing issues, including template issues, issue
	classification, etc.?
  \item Use of CI/CD
\end{itemize}

\section{Project Decomposition and Scheduling}

\begin{itemize}
  \item How will you be using GitHub projects?
  \item Include a link to your GitHub project
\end{itemize}

\wss{How will the project be scheduled?  This is the big picture schedule, not
details. You will need to reproduce information that is in the course outline
for deadlines.}

\section{Proof of Concept Demonstration Plan}

What is the main risk, or risks, for the success of your project?  What will you
demonstrate during your proof of concept demonstration to convince yourself that
you will be able to overcome this risk?

\section{Expected Technology}

This project will have a machine learning aspect, along with a user interface or dashboard, and a data processing pipeline. The technologies below were selected initially, however, there will be changes as the project progresses due to stakeholder constraints and future design choices.

\begin{table}[H]
\caption{Languages and Libraries} \label{TblLanguagesAndLibraries}
\begin{tabularx}{\textwidth}{p{3.5cm}p{7cm}}
\toprule
\textbf{Technology} & \textbf{Explanation}\\
\midrule
Python & Most widely-used language for machine learning and dashboard development.\\
\addlinespace
NumPy, Pandas, Matplotlib & Common libraries used for data manipulation visualization.\\
\bottomrule
\end{tabularx}
\end{table}

\begin{table}[!htbp]
\caption{Machine Learning} \label{TblMachineLearning}
\begin{tabularx}{\textwidth}{p{3.5cm}p{7cm}}
\toprule
\textbf{Technology} & \textbf{Explanation}\\
\midrule
PyTorch, TensorFlow & Widely used frameworks for machine learning.\\
\addlinespace
YOLOv8, Faster R-CNN, DETR & Object detection architectures to be used for custom detectors.\\
\addlinespace
OpenCV & Computer vision library. Includes object detection algorithms.\\
\addlinespace
ImageNet-1000 & Dataset containing 1000 different categories. May be a good starting point for cow detection.\\
\bottomrule
\end{tabularx}
\end{table}

\begin{table}[!htbp]
\caption{Storage and Environment} \label{TblStorageAndEnvironment}
\begin{tabularx}{\textwidth}{p{3cm}p{7.5cm}}
\toprule
\textbf{Technology} & \textbf{Explanation}\\
\midrule
PostgreSQL & To store data on cow behaviour and/or attributes.\\
\addlinespace
Docker & Allows us to build and deploy our application.\\
\bottomrule
\end{tabularx}
\end{table}

\begin{table}[!htbp]
\caption{Development Tools and Testing} \label{TblDevelopmentToolsAndTesting}
\begin{tabularx}{\textwidth}{p{3cm}p{7cm}}
\toprule
\textbf{Technology} & \textbf{Explanation}\\
\midrule
Git & For version control and organization.\\
\addlinespace
GitHub Projects, GitHub Issues & For project tracking and task delegation.\\
\addlinespace
GitHub Actions & May be used for CI/CD pipeline.\\
\addlinespace
Pytest & Python testing framework to be used in our CI/CD pipeline.\\
\bottomrule
\end{tabularx}
\end{table}

\begin{table}[!htbp]
\caption{Front-End} \label{TblFrontEnd}
\begin{tabularx}{\textwidth}{p{3cm}p{7cm}}
\toprule
\textbf{Technology} & \textbf{Explanation}\\
\midrule
Streamlit or React.js & Web-based Front-end options based on the desired level of complexity in our UI.\\
\addlinespace
PyQt/PySide & Option for a desktop standalone app.\\
\bottomrule
\end{tabularx}
\end{table}


\section{Coding Standard}

\wss{What coding standard will you adopt?}

\newpage{}

\section*{Appendix --- Reflection}

\wss{Not required for CAS 741}

The purpose of reflection questions is to give you a chance to assess your own
learning and that of your group as a whole, and to find ways to improve in the
future. Reflection is an important part of the learning process.  Reflection is
also an essential component of a successful software development process.  

Reflections are most interesting and useful when they're honest, even if the
stories they tell are imperfect. You will be marked based on your depth of
thought and analysis, and not based on the content of the reflections
themselves. Thus, for full marks we encourage you to answer openly and honestly
and to avoid simply writing ``what you think the evaluator wants to hear.''

Please answer the following questions.  Some questions can be answered on the
team level, but where appropriate, each team member should write their own
response:


\begin{enumerate}
    \item Why is it important to create a development plan prior to starting the
    project?
    \item In your opinion, what are the advantages and disadvantages of using
    CI/CD?
    \item What disagreements did your group have in this deliverable, if any,
    and how did you resolve them?
\end{enumerate}

\newpage{}

\section*{Appendix --- Team Charter}

\wss{borrows from
\href{https://engineering.up.edu/industry_partnerships/files/team-charter.pdf}
{University of Portland Team Charter}}

\subsection*{External Goals}
Our team hopes to have a positive impact on the targeted users of this product by making their day-to-day tasks easier so they can focus on what matters most on their farms. We also want to gain practical and hands-on experience in the machine learning domain. Artificial Intelligence is a field we all find fascinating, but have yet to explore in depth. Lastly, we look forward to making this a positive learning experience where we can all be proud of the work we produce by the end of the year.

\subsection*{Attendance}

\subsubsection*{Expectations}

\wss{What are your team's expectations regarding meeting attendance (being on
time, leaving early, missing meetings, etc.)?}

\subsubsection*{Acceptable Excuse}

\wss{What constitutes an acceptable excuse for missing a meeting or a deadline?
What types of excuses will not be considered acceptable?}

\subsubsection*{In Case of Emergency}

\wss{What process will team members follow if they have an emergency and cannot
attend a team meeting or complete their individual work promised for a team
deliverable?}

\subsection*{Accountability and Teamwork}

\subsubsection*{Quality} 

\wss{What are your team's expectations regarding the quality
of team members' preparation for team meetings and the quality of the
deliverables that members bring to the team?}

\subsubsection*{Attitude}

We want to create an environment where all ideas are welcome and creative thinking is encouraged. Below is a list of \textbf{team expectations:}

\begin{enumerate}
    \item \textbf{Professionalism:} Team members should deliver high quality work and remain punctual.
    \item \textbf{Respect:} All ideas will be given fair attention and team members will respect opposing ideas.
    \item \textbf{Cooperation:} Everyone on the team is expected to contribute equally to project deliverables and come to meetings prepared.
    \item \textbf{Attitude:} Everyone on the team should maintain a positive, and encouraging attitude to maintain productivity and a low-stress environment.
\end{enumerate}

\noindent In the case that any conflict arises between team members, the following \textbf{conflict resolution plan} will be executed:

\begin{enumerate}
    \item \textbf{Communication:} the first stage of conflict resolution is to communicate the issue one-on-one with the involved team member to define the problem.
    \item \textbf{Find a middle ground:} The involved team members should try to understand each other's perspectives and aim to reach a positive solution.
    \item \textbf{Escalate:} In the case a middle ground cannot be reached, the team will escalate and communicate the issue with Dr. Smith.
\end{enumerate}


\subsubsection*{Stay on Track}

\wss{What methods will be used to keep the team on track? How will your team
ensure that members contribute as expected to the team and that the team
performs as expected? How will your team reward members who do well and manage
members whose performance is below expectations?  What are the consequences for
someone not contributing their fair share?}

\wss{You may wish to use the project management metrics collected for the TA and
instructor for this.}

\wss{You can set target metrics for attendance, commits, etc.  What are the
consequences if someone doesn't hit their targets?  Do they need to bring the
coffee to the next team meeting?  Does the team need to make an appointment with
their TA, or the instructor?  Are there incentives for reaching targets early?}

\subsubsection*{Team Building}

\wss{How will you build team cohesion (fun time, group rituals, etc.)? }

\subsubsection*{Decision Making} 

\wss{How will you make decisions in your group? Consensus?  Vote? How will you
handle disagreements? }

\end{document}