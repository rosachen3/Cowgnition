\documentclass{article}
\usepackage{float}
\usepackage{booktabs}
\usepackage{tabularx}
\usepackage{enumitem}

\title{Development Plan\\\progname}

\author{\authname}

\date{}

\input{../Comments}
%% Common Parts

\newcommand{\progname}{Software Engineering} % PUT YOUR PROGRAM NAME HERE
\newcommand{\authname}{Team 24, Cowgnition
\\ Sylvia Kamel
\\ Rosa Chen
\\ Karim Elbasiouni
\\ Claire Nielsen
\\ Safwan Khan} % AUTHOR NAMES                  

\usepackage{hyperref}
    \hypersetup{colorlinks=true, linkcolor=blue, citecolor=blue, filecolor=blue,
                urlcolor=blue, unicode=false}
    \urlstyle{same}
                                

\begin{document}

\maketitle

\begin{table}[hp]
\caption{Revision History} \label{TblRevisionHistory}
\begin{tabularx}{\textwidth}{llX}
\toprule
\textbf{Date} & \textbf{Developer(s)} & \textbf{Change}\\
\midrule
Date1 & Name(s) & Description of changes\\
Date2 & Name(s) & Description of changes\\
... & ... & ...\\
\bottomrule
\end{tabularx}
\end{table}

\newpage{}

\wss{Put your introductory blurb here.  Often the blurb is a brief roadmap of
what is contained in the report.}

\wss{Additional information on the development plan can be found in the
\href{https://gitlab.cas.mcmaster.ca/courses/capstone/-/blob/main/Lectures/L02b_POCAndDevPlan/POCAndDevPlan.pdf?ref_type=heads}
{lecture slides}.}

\section{Confidential Information?}

\wss{State whether your project has confidential information from industry, or
not.  If there is confidential information, point to the agreement you have in
place.}

\wss{For most teams this section will just state that there is no confidential
information to protect.}
\section{IP to Protect}

\wss{State whether there is IP to protect.  If there is, point to the agreement.
All students who are working on a project that requires an IP agreement are also
required to sign the ``Intellectual Property Guide Acknowledgement.''}

\section{Copyright License}

\wss{What copyright license is your team adopting.  Point to the license in your
repo.}

\section{Team Meeting Plan}

\wss{How often will you meet? where?}

\wss{If the meeting is a physical location (not virtual), out of an abundance of
caution for safety reasons you shouldn't put the location online}

\wss{How often will you meet with your industry advisor?  when?  where?}

\wss{Will meetings be virtual?  At least some meetings should likely be
in-person.}

\wss{How will the meetings be structured?  There should be a chair for all meetings.  There should be an agenda for all meetings.}

\section{Team Communication Plan}

\wss{Issues on GitHub should be part of your communication plan.}

\section{Team Member Roles}

For the remainder of this project, we will work collaboratively as a team following assumed roles to develop the Cow-puter Vision project. These defined roles may have different assignees at different stages of the project, as team members may wish to rotate between the roles.
The table below contains a list of all the roles and respective descriptions that are believed to be essential to ensure successful collaboration and achievement of the project goals:

\subsection{Non-technical Roles}
\begin{table}[ht]
\centering
\caption{Non-technical roles and descriptions}
\label{tab:non-technical-roles}
\begin{tabularx}{\textwidth}{lX}
\toprule
\textbf{Role} & \textbf{Description} \\
\midrule

Team Liaison & Acts as the main point of contact between the team and supervisors and stakeholders.  \\ \\
Notetaker & Records meeting minutes, lecture notes, and sets agenda. \\ \\
Meeting Chair & Facilitates team discussions, ensuring that the team remains focused on the meeting agenda without going off on tangents. \\ \\
Backlog Manager & Creates, tracks, and assigns project-related GitHub issues based on deliverable sections and sets their priority based on deadlines and business value.\\ \\
\bottomrule
\end{tabularx}
\end{table}

\noindent\textbf{Important Note:} Backlog Manager is not soley responsible for creating GitHub issues, and is only responsible for creating issues based on the deliverable sections. Example: Backlog
Manager is responsible for creating issues such as "Complete section 1.1" and "Review section 1.1". Issues involving very specific tasks, 
that are not clearly outlined or sectioned on deliverable templates, such as fixing specific bugs, fall under the responsibility of 
the team member(s) who encounters the issue or task to begin with.\\

\subsection{Technical Roles}
\begin{table}[H]
\centering
\caption{Technical roles and descriptions}
\label{tab:technical-roles}
\begin{tabularx}{\textwidth}{lX}
\toprule
\textbf{Role} & \textbf{Description} \\
\midrule

Computer Vision \& ML Engineer & Responsible for leading design and development of the ML model, including model training and evaluation of the computer vision models using the preprocessed cow data, as well as ensuring that the model achieves satisfactory performance.\\ \\
Data Pipeline Engineer &  Responsible for leading the collection, organization and preprocessing of the raw image/video data, ensuring it is properly labeled and cleaned, and preparing it into standardized datasets ready for model training and evaluation.\\ \\
Backend \& Integration Engineer  & Responsible for leading the development of APIs or services that connect the trained model to the frontend, and manages a database to store any necessary data, ensuring smooth communication between components and managing data flow through the system \\ \\
Frontend \& UI/UX Engineer  & Responsible for leading the design and development of interface for farmers to interact with, and ensure that the necessary results are clearly displayed to them. \\ \\
Deployment \& Testing Lead & Responsible for ensuring that the model deploys to the target environment, testing and validating the accuracy and reliability of the system, running end-to-end tests, and fixing integration issues\\ \\
\bottomrule
\end{tabularx}
\end{table}


\section{Workflow Plan}

\begin{itemize}
	\item How will you be using git, including branches, pull request, etc.?
	\item How will you be managing issues, including template issues, issue
	classification, etc.?
  \item Use of CI/CD
\end{itemize}

\section{Project Decomposition and Scheduling}

\begin{itemize}
  \item How will you be using GitHub projects?
  \item Include a link to your GitHub project
\end{itemize}

\wss{How will the project be scheduled?  This is the big picture schedule, not
details. You will need to reproduce information that is in the course outline
for deadlines.}

\section{Proof of Concept Demonstration Plan}

What is the main risk, or risks, for the success of your project?  What will you
demonstrate during your proof of concept demonstration to convince yourself that
you will be able to overcome this risk?

\section{Expected Technology}

\wss{What programming language or languages do you expect to use?  What external
libraries?  What frameworks?  What technologies.  Are there major components of
the implementation that you expect you will implement, despite the existence of
libraries that provide the required functionality.  For projects with machine
learning, will you use pre-trained models, or be training your own model?  }

\wss{The implementation decisions can, and likely will, change over the course
of the project.  The initial documentation should be written in an abstract way;
it should be agnostic of the implementation choices, unless the implementation
choices are project constraints.  However, recording our initial thoughts on
implementation helps understand the challenge level and feasibility of a
project.  It may also help with early identification of areas where project
members will need to augment their training.}

Topics to discuss include the following:

\begin{itemize}
\item Specific programming language
\item Specific libraries
\item Pre-trained models
\item Specific linter tool (if appropriate)
\item Specific unit testing framework
\item Investigation of code coverage measuring tools
\item Specific plans for Continuous Integration (CI), or an explanation that CI
  is not being done
\item Specific performance measuring tools (like Valgrind), if
  appropriate
\item Tools you will likely be using?
\end{itemize}

\wss{git, GitHub and GitHub projects should be part of your technology.}

\section{Coding Standard}

\wss{What coding standard will you adopt?}

\newpage{}

\section*{Appendix --- Reflection}

\wss{Not required for CAS 741}

\input{../Reflection.tex}

\begin{enumerate}
    \item Why is it important to create a development plan prior to starting the
    project?
    \item In your opinion, what are the advantages and disadvantages of using
    CI/CD?
    \item What disagreements did your group have in this deliverable, if any,
    and how did you resolve them?
\end{enumerate}

\newpage{}

\section*{Appendix --- Team Charter}

\wss{borrows from
\href{https://engineering.up.edu/industry_partnerships/files/team-charter.pdf}
{University of Portland Team Charter}}

\subsection*{External Goals}

\wss{What are your team's external goals for this project? These are not the
goals related to the functionality or quality fo the project.  These are the
goals on what the team wishes to achieve with the project.  Potential goals are
to win a prize at the Capstone EXPO, or to have something to talk about in
interviews, or to get an A+, etc.}

\subsection*{Attendance}

\subsubsection*{Expectations}

\wss{What are your team's expectations regarding meeting attendance (being on
time, leaving early, missing meetings, etc.)?}

\subsubsection*{Acceptable Excuse}

\wss{What constitutes an acceptable excuse for missing a meeting or a deadline?
What types of excuses will not be considered acceptable?}

\subsubsection*{In Case of Emergency}

\wss{What process will team members follow if they have an emergency and cannot
attend a team meeting or complete their individual work promised for a team
deliverable?}

\subsection*{Accountability and Teamwork}

\subsubsection*{Quality} 

\wss{What are your team's expectations regarding the quality
of team members' preparation for team meetings and the quality of the
deliverables that members bring to the team?}

\subsubsection*{Attitude}

\wss{What are your team's expectations regarding team members' ideas,
interactions with the team, cooperation, attitudes, and anything else regarding
team member contributions?  Do you want to introduce a code of conduct?  Do you
want a conflict resolution plan?  Can adopt existing codes of conduct.}

\subsubsection*{Stay on Track}

\noindent\textbf{Meeting Deadlines:} The team will employ the use of the official capstone calendar to keep track of dates and deadlines. Each team member is responsible for ensuring that they aware of
upcoming deadlines and potential postponement of these deadlines. Regardless, the team will collectively make an effort to communicate any reminders regarding dates, deadlines,
and any announcements made my the professor through the team's private Microsoft Team's channel. \\

\noindent\textbf{Tracking Progress:} GitHub issues will be created to split milestone tasks to ensure that the volume of work is evenly distributed. Brief progress check-ins will be done at the beginning of every weekly meeting, through which every 
team member is required to share their current progress on their assigned tasks, including any concerns or difficulties that they may have encountered. The team reserves the right to adjust division of responsibilities if a task appears to be
more difficult to complete than expected, or to accomodate for another team member's conflicting schedules, such as in the case when a team member may have external committments or circumstances that other team members do not share. Examples include
conflicting midterm schedules, important medical appointments, etc. \\

\noindent\textbf{Rewards for High Performance:} Team members who perform beyond their expectations will be celebrated in team meetings and given the recognition they rightfully deserve, as well as assigned leadership roles in future milestones that better reflect their
committment to the team. Following the conclusion of the project, the team may pitch in and buy the high performing individual any candy of their choice. Moreover, the team will make a collective request to the professor that the performing individual is given a bonus grade,
though this may or may not be accepted by the professor. \\

\noindent\textbf{Dealing with Unmet Expectations:}
If no extenuating circumstances are present, and a team member is not meeting their workload committments for longer than two weeks or up to the closest milestone, then the following procedure will be followed to address unmet expectations: \\

\begin{enumerate}
  \item Unmet expectations will be addressed briefly as the first item on the agenda of the closest upcoming meeting. If the issue persists, or the team member does not attend the meeting
  proceed to step 2.
  \item The team will have an in-person meeting hosted before, during (if lecture slot is empty), or after lecture to discuss with the team member in question. If the issue persists, or the team member does not attend the meeting
  proceed to step 3.
  \item The team liaison will contact the TA to seek advice on handling the sitaution. If after applying the TA's advice the issue still persists, proceed to step 4.
  \item The team liaison will contact the professor to address the situation.
\end{enumerate}

\noindent\textbf{Consequences for Under-performing:}
The consequences for not meeting the expected workload committment include: 
\begin{itemize}
  \item Negative peer evaluation
  \item Loss of grades
  \item Academic dishonesty punishments
\end{itemize}


\noindent\textbf{Incentives for Meeting Deadlines Early:} Team members who meet deadlines early will be given the option
to have a lighter workload for the next deliverable or milestone. This does not apply for the winter semester,
as the team expects to have a tougher workload around that time period, due to final implementation deadlines. \\

\noindent\textbf{Target Metrics:}
Each team member is responsible to achieve the following metrics to show their contribution:
\begin{itemize}
  \item Attend 100\% of all meetings (except if there is an acceptable excuse)
  \item Confirm review of all documents prior to submission, ensuring adherence to rubric
  \item Contribute to roughly 20\% of the total volume of work, tracked through weekly number of closed issues.
\end{itemize}


\subsubsection*{Team Building}

\wss{How will you build team cohesion (fun time, group rituals, etc.)? }

\subsubsection*{Decision Making} 

\wss{How will you make decisions in your group? Consensus?  Vote? How will you
handle disagreements? }

\end{document}